\begin{table}[!ht]
\caption{New GRB redshifts\label{tab:newz}}
\centering
\begin{tabular}{ccc}
\hline
\hline\noalign{\smallskip}
{GRB host} & {Redshift} & Spectral features \\
\hline\noalign{\smallskip}
GRB~050714B & 2.4383  & \oiii(5007), \ha  \\
GRB~060306\tablefootmark{a}  & 1.5585 / 1.5597 & \oii, \ha \\
GRB~060805A\tablefootmark{b}  & 2.3633 & \hb, \oiii(5007), \ha \\
GRB~060923B & 1.5094 & \oiii(5007), \ha \\
GRB~070224  & 1.9922 & \oiii(4959,5007) \\
GRB~070328  & 2.0627 & \oii, \hb, \oiii \\
\noalign{\smallskip}\hline\noalign{\smallskip}
GRB~070521  & 2.0865 & \ha \\
GRB~080602  & 1.8204 & \oii, \oiii, \ha \\
GRB~090201  & 2.1000 & \oii, \hb, \oiii, \ha \\
\noalign{\smallskip}\hline\noalign{\smallskip}
%GRB~081210  & 2.0631 & \oii, \hb, \oiii \\
GRB~100508A  & 0.5201 & \oii, \hb, \oiii, \ha \\
GRB~111129A & 1.0796 & \oii \\
GRB~120211A & $2.4\pm0.1$ & Ly-$\alpha$ break \\
GRB~120224A & $1.1\pm0.2$ & Balmer break \\
GRB~120805A & $3.1\pm0.1$ & Ly-$\alpha$ break \\
GRB~121209A & $2.1\pm0.3$ & Balmer break \\
GRB~140114A & $3.0\pm0.1$ & Ly-$\alpha$ break \\
\hline\noalign{\smallskip}

\end{tabular}
\tablefoot{The horizontal lines denote the separation between TOUGH GRBs (upper part), BAT6 (middle part) and others. 
\tablefoottext{a}{Emission lines of the host of GRB~060306 have a double-peaked profile, and we thus provide two redshifts.}
\tablefoottext{b}{Ambiguous association because two galaxies are present in the XRT error-circle. This is the redshift for the fainter of the two objects, for details see \citet{2012ApJ...752...62J}.}} 
\end{table}