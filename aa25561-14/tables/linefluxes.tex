\begin{longtab}
%\begin{landscape}
\begin{longtable}{c c c c c c}
\caption{Fluxes of Balmer lines for GRB hosts\label{tab:balmerlines}}\\
\hline\hline
{GRB host} & {Redshift} & $\rm{H}\delta$ & $\rm{H}\gamma$ & $\rm{H}\beta$ & $\rm{H}\alpha$ \\
\hline
\endfirsthead
\hline
\caption{Balmer lines fluxes (continued)}\\
\hline\hline
{GRB host} & {Redshift} & $\rm{H}\delta$ & $\rm{H}\gamma$ & $\rm{H}\beta$ & $\rm{H}\alpha$ \\
\hline
\endhead
%GRB000210 & 0.8456 & $0.2 \pm 0.1 \pm 0.1$& $0.9 \pm 0.1 \pm 0.1$& $1.5 \pm 0.1 \pm 0.2$& $6.8 \pm 0.5 \pm 0.9$\\ 
%GRB000418 & 1.1181 & $1.4 \pm 0.3 \pm 0.1$& $1.2 \pm 0.3 \pm 0.1$& $2.8 \pm 5.3 \pm 0.5$& \nodata \\ 
%GRB000911 & 1.0578 & $0.2 \pm 0.1 \pm 0.1$& $0.2 \pm 0.1 \pm 0.1$& $0.7 \pm 0.4 \pm 0.2$& \nodata \\ 
%GRB011211 & 2.1435 & $0.3 \pm 0.7 \pm 0.3$& \nodata & $0.7 \pm 0.6 \pm 0.3$& $3.5 \pm 1.3 \pm 1.2$\\ 
%GRB021004 & 2.3296 & \nodata & \nodata & $2.6 \pm 0.4 \pm 0.4$& $9.5 \pm 1.2 \pm 1.3$\\ 
%GRB050219A & 0.2114 & $-3.9 \pm 1.0 \pm 0.1$& $-3.0 \pm 1.0 \pm 0.1$& \nodata & $-4.0 \pm 1.3 \pm 0.1$\\ 
GRB050416A & 0.6542 & $0.3 \pm 0.3 \pm 0.1$& $1.4 \pm 0.3 \pm 0.2$& $2.9 \pm 0.4 \pm 0.3$& $16.0 \pm 2.0 \pm 1.6$\\ 
GRB050525A & 0.6063 & $0.1 \pm 0.1 \pm 0.1$& $0.1 \pm 0.1 \pm 0.1$& $0.3 \pm 0.2 \pm 0.1$& $0.7 \pm 0.2 \pm 0.3$\\ 
GRB050714B & 2.4383 & \nodata & $0.6 \pm 0.3 \pm 0.2$& $0.5 \pm 0.2 \pm 0.2$& $3.4 \pm 0.6 \pm 0.8$\\ 
GRB050819A & 2.5042 & \nodata & $0.6 \pm 0.3 \pm 0.1$& $1.0 \pm 0.3 \pm 0.2$& \nodata \\ 
GRB050824 & 0.8277 & $0.7 \pm 0.1 \pm 0.1$& $1.1 \pm 0.1 \pm 0.2$& $2.6 \pm 0.4 \pm 0.5$& $6.8 \pm 0.6 \pm 1.3$\\ 
GRB050915A & 2.5275 & \nodata & $0.5 \pm 0.2 \pm 0.1$& $1.8 \pm 0.2 \pm 0.4$& \nodata \\ 
GRB051001 & 2.4295 & \nodata & $1.3 \pm 0.6 \pm 0.3$& $1.9 \pm 0.3 \pm 0.4$& $13 \pm 4 \pm 2$\\ 
GRB051016B & 0.9358 & $2.4 \pm 0.3 \pm 0.2$ & $6.3 \pm 0.4 \pm 0.5$ & $10.9 \pm 1.2 \pm 1.1$ & $40 \pm 6 \pm 4$\\ 
GRB051022A & 0.8061 & $2.8 \pm 0.5 \pm 0.3$& $8.1 \pm 0.8 \pm 0.7$& $22.1 \pm 1.3 \pm 1.7$& $110 \pm 4 \pm 15$\\ 
GRB051117B & 0.4805 & $-1.4 \pm 0.8 \pm 0.1$& $1.1 \pm 1.0 \pm 0.1$& $3.0 \pm 1.0 \pm 0.2$& $21.1 \pm 4.1 \pm 1.3$\\ 
% GRB 060204B added by hand during refereeing
GRB060204B & 2.3393 & \nodata & \nodata & $4.1 \pm 1.1 \pm 0.6$& $16.9 \pm 1.4 \pm 2.0$ \\ 
GRB060306 & 1.5597 & \nodata & \nodata & $1.4 \pm 2.0 \pm 0.4$& $8.9 \pm 3.4 \pm 1.4$\\ 
GRB060604 & 2.1355 & \nodata & \nodata & $0.4 \pm 0.1 \pm 0.1$& $1.8 \pm 0.3 \pm 0.5$\\ 
GRB060707 & 3.4246 & \nodata & \nodata & $1.1 \pm 1.8 \pm 0.8$& \nodata \\ 
GRB060719 & 1.5318 & \nodata & $0.1 \pm 0.5 \pm 0.1$& $0.8 \pm 0.5 \pm 0.2$& $3.8 \pm 0.4 \pm 0.7$\\ 
GRB060729 & 0.5429 & \nodata & $0.0 \pm 0.3 \pm 0.1$& $0.6 \pm 0.2 \pm 0.1$& $3.7 \pm 1.7 \pm 0.5$\\ 
GRB060805A & 2.3633 & \nodata & $0.5 \pm 0.4 \pm 0.2$& $1.4 \pm 0.3 \pm 0.4$& $3.7 \pm 0.5 \pm 0.9$\\ 
GRB060814 & 1.9223 & \nodata & \nodata & $8.3 \pm 3.0 \pm 1.5$& $28 \pm 4 \pm 4$\\ 
GRB060912A & 0.9362 & $0.5 \pm 0.3 \pm 0.1$& $2.5 \pm 0.3 \pm 0.2$& $5.9 \pm 0.5 \pm 0.5$& $15.8 \pm 3.8 \pm 1.6$\\ 
GRB060923B & 1.5094 & \nodata & \nodata & $-1.6 \pm 0.6 \pm 0.1$& $2.6 \pm 0.8 \pm 0.3$\\ 
GRB060926 & 3.2090 & $0.1 \pm 0.1 \pm 0.1$& \nodata & $0.3 \pm 0.2 \pm 0.2$& \nodata \\ 
GRB061021 & 0.3453 & $0.2 \pm 0.2 \pm 0.1$& $0.2 \pm 0.2 \pm 0.1$& $0.5 \pm 0.1 \pm 0.1$& $1.9 \pm 0.1 \pm 0.2$\\ 
GRB061110A & 0.7578 & $0.0 \pm 0.1 \pm 0.1$& \nodata & $0.3 \pm 0.1 \pm 0.1$& \nodata \\ 
% GRB 061202 added by hand during refereeing
GRB061202 & 2.2543 & \nodata & \nodata & $1.2 \pm 0.4 \pm 0.3$& $6.1 \pm 0.8 \pm 1.3$ \\ 
GRB070103 & 2.6208 & $1.1 \pm 1.9 \pm 0.6$& $3.9 \pm 1.6 \pm 1.2$& $4.3 \pm 0.8 \pm 1.1$& \nodata \\ 
GRB070110 & 2.3523 & \nodata & $0.6 \pm 0.4 \pm 0.2$& \nodata & $3.5 \pm 0.6 \pm 0.8$\\ 
GRB070129 & 2.3384 & \nodata & $0.7 \pm 0.6 \pm 0.2$& $1.5 \pm 0.9 \pm 0.4$& $6.5 \pm 1.1 \pm 1.1$\\ 
GRB070224 & 1.9922 & \nodata & $0.1 \pm 0.1 \pm 0.1$& $0.2 \pm 0.1 \pm 0.2$& \nodata \\ 
GRB070306 & 1.4965 & $1.7 \pm 1.3 \pm 0.1$& $7.7 \pm 3.6 \pm 0.8$& $11.6 \pm 1.2 \pm 0.7$& $53.5 \pm 1.4 \pm 3.7$\\ 
GRB070318 & 0.8401 & \nodata & $0.3 \pm 0.1 \pm 0.1$& $1.2 \pm 0.2 \pm 0.2$& $3.3 \pm 0.4 \pm 0.5$\\ 
GRB070328 & 2.0627 & $0.3 \pm 0.2 \pm 0.1$& $0.1 \pm 0.3 \pm 0.1$& $1.1 \pm 0.2 \pm 0.2$& \nodata \\ 
GRB070419B & 1.9586 & \nodata & \nodata & $0.9 \pm 0.3 \pm 0.3$& $4.4 \pm 0.6 \pm 1.1$\\ 
GRB070521 & 2.0865 & \nodata & \nodata & $-0.2 \pm 0.9 \pm 0.2$& $10 \pm 5 \pm 3$\\ 
GRB070802 & 2.4538 & \nodata & $0.4 \pm 0.2 \pm 0.1$& $1.4 \pm 0.2 \pm 0.3$& $5.1 \pm 0.7 \pm 1.0$\\ 
GRB071021 & 2.4515 & \nodata & $0.7 \pm 0.5 \pm 0.2$& $2.8 \pm 0.3 \pm 0.5$& $8.9 \pm 1.8 \pm 1.8$\\ 
%GRB071117 & 1.3293 & \nodata & \nodata & $2.3 \pm 1.3 \pm 0.8$& $9 \pm 2 \pm 3$\\ 
GRB080207 & 2.0856 & \nodata & $1.0 \pm 0.6 \pm 0.3$& $1.3 \pm 0.5 \pm 0.3$& $10.9 \pm 1.9 \pm 2.2$\\ 
GRB080413B & 1.1012 & \nodata & $0.2 \pm 0.1 \pm 0.1$& \nodata & $2.6 \pm 1.0 \pm 0.8$\\ 
%GRB080520 & 1.5467 & $1.7 \pm 0.9 \pm 0.3$& $2.5 \pm 0.8 \pm 0.4$& $9.5 \pm 0.7 \pm 1.3$& $36.1 \pm 1.3 \pm 4.7$\\ 
GRB080602 & 1.8204 & $1.6 \pm 0.9 \pm 0.3$& $2.1 \pm 0.7 \pm 0.3$& \nodata & $30 \pm 8 \pm 4$\\ 
GRB080605 & 1.6408 & \nodata & \nodata & $7.7 \pm 0.8 \pm 1.3$& $29.1 \pm 1.4 \pm 4.3$\\ 
GRB080804 & 2.2059 & \nodata & \nodata & $0.9 \pm 0.4 \pm 0.3$& $3.7 \pm 0.8 \pm 1.2$\\ 
GRB080805 & 1.5052 & \nodata & \nodata & $1.7 \pm 0.6 \pm 0.5$& $11.0 \pm 1.6 \pm 2.7$\\ 
GRB081109 & 0.9785 & \nodata & $1.8 \pm 0.4 \pm 0.3$ & $5.5 \pm 0.5 \pm 0.8$& $20.9 \pm 0.9 \pm 2.7$\\ 
GRB081210 & 2.0631 & $0.3 \pm 0.3 \pm 0.1$& $1.1 \pm 0.4 \pm 0.3$& $2.1 \pm 0.3 \pm 0.5$& \nodata \\ 
GRB081221 & 2.2590 & \nodata & \nodata & $2.4 \pm 1.3 \pm 1.7$& $9.5 \pm 1.9 \pm 5.3$\\ 
GRB090113 & 1.7494 & \nodata & $3.6 \pm 1.2 \pm 0.7$& $3.3 \pm 0.9 \pm 0.6$& $15 \pm 2 \pm 3$\\ 
GRB090201 & 2.1000 & \nodata & $3.8 \pm 1.8 \pm 1.1$& $6.0 \pm 1.4 \pm 1.4$& $23 \pm 2 \pm 4$\\ 
GRB090323 & 3.5832 & \nodata & $-0.3 \pm 0.4 \pm 0.1$& $0.0 \pm 0.5 \pm 0.1$& \nodata \\ 
GRB090407 & 1.4478 & $-0.2 \pm 0.3 \pm 0.1$& $-0.1 \pm 0.3 \pm 0.1$& $0.7 \pm 0.2 \pm 0.1$& $4.4 \pm 0.4 \pm 0.6$\\ 
GRB090926B & 1.2427 & $0.5 \pm 0.2 \pm 0.1$& $0.7 \pm 0.4 \pm 0.2$& $2.7 \pm 0.9 \pm 0.5$& $14.2 \pm 2.5 \pm 2.4$\\ 
GRB091018 & 0.9710 & \nodata & $0.0 \pm 0.2 \pm 0.1$& $1.5 \pm 0.8 \pm 0.1$& $4.2 \pm 0.6 \pm 0.3$\\ 
GRB091127 & 0.4904 & $0.7 \pm 0.2 \pm 0.1$& $0.7 \pm 0.2 \pm 0.1$& $1.6 \pm 0.2 \pm 0.1$& $5.5 \pm 0.3 \pm 0.5$\\ 
%GRB100316B & 1.1809 & \nodata & $0.0 \pm 0.2 \pm 0.1$& $0.0 \pm 0.7 \pm 0.1$& \nodata \\ 
GRB100316D & 0.0592 & $14.0 \pm 0.7$& $27.8 \pm 1.4$& $62 \pm 3$& $212 \pm 11$\\ 
GRB100418A & 0.6235 & $1.6 \pm 1.3 \pm 0.3$& $4.5 \pm 0.5 \pm 0.5$& $10.0 \pm 0.7 \pm 1.0$& $35 \pm 3 \pm 4$\\ 
GRB100424A & 2.4656 & \nodata & $0.8 \pm 0.4 \pm 0.3$& $2.0 \pm 0.2 \pm 0.5$& $6.5 \pm 1.6 \pm 1.9$\\ 
GRB100508A & 0.5201 & \nodata & $1.0 \pm 0.9 \pm 0.1$& $7.2 \pm 0.6 \pm 0.5$& $24.9 \pm 1.8 \pm 1.6$\\ 
% GRB 100606A added by hand during refereeing
GRB100606A & 1.5545 & \nodata & \nodata & $1.7 \pm 1.0 \pm 0.7$& $4.7 \pm 0.5 \pm 1.4$ \\ 
GRB100615A & 1.3978 & \nodata & $0.2 \pm 0.4 \pm 0.1$& $1.0 \pm 0.4 \pm 0.2$& $4.8 \pm 0.5 \pm 0.7$\\ 
GRB100621A & 0.5426 & \nodata & $18.8 \pm 0.8 \pm 1.1$& $43.8 \pm 1.0 \pm 2.4$& $128 \pm 5 \pm 7$\\ 
GRB100724A & 1.2890 & \nodata & $0.2 \pm 0.3 \pm 0.1$& $1.1 \pm 0.7 \pm 0.5$& $3.8 \pm 0.5 \pm 1.1$\\ 
GRB100728A & 1.5670 & \nodata & $1.2 \pm 2.5 \pm 1.4$& $2.5 \pm 2.5 \pm 1.9$& $10.6 \pm 1.7 \pm 4.7$\\ 
GRB100814A & 1.4392 & \nodata & \nodata & $1.3 \pm 0.3 \pm 0.2$& $4.0 \pm 0.3 \pm 0.4$\\ 
GRB100816A & 0.8048 & \nodata & \nodata & $1.7 \pm 0.4 \pm 0.2$& $19 \pm 2 \pm 3$\\ 
%GRB110731A & 2.8308 & \nodata & \nodata & $0.7 \pm 0.6 \pm 0.3$& \nodata \\ 
GRB110808A & 1.3490 & \nodata & \nodata & $2.0 \pm 0.6 \pm 0.4$& $7.6 \pm 0.5 \pm 1.1$\\ 
GRB110818A & 3.3609 & \nodata & \nodata & $1.6 \pm 0.4 \pm 0.3$& \nodata \\ 
GRB110918A & 0.9843 & \nodata & $-0.1 \pm 1.6 \pm 0.1$& $10.3 \pm 2.4 \pm 0.6$& $42 \pm 11 \pm 4$\\ 
GRB111123A & 3.1513 & \nodata & \nodata & \nodata & \nodata \\ 
GRB111129A & 1.0796 & \nodata & $1.1 \pm 0.5 \pm 0.1$& $3.3 \pm 1.9 \pm 0.4$& \nodata \\ 
GRB111209A & 0.6770 & $0.2 \pm 0.1 \pm 0.1$& $0.3 \pm 0.1 \pm 0.1$& $0.7 \pm 0.1 \pm 0.1$& $2.4 \pm 0.6 \pm 0.5$\\ 
GRB111211A & 0.4786 & \nodata & $0.5 \pm 0.7 \pm 0.1$& $-0.5 \pm 0.4 \pm 0.1$& $2.3 \pm 0.4 \pm 0.2$\\ 
GRB111228A & 0.7164 & \nodata & $0.1 \pm 0.2 \pm 0.1$& $0.5 \pm 0.3 \pm 0.1$& \nodata \\ 
GRB120118B & 2.9428 & \nodata & $2.0 \pm 0.3 \pm 0.3$& $2.4 \pm 0.8 \pm 0.3$& \nodata \\ 
GRB120119A & 1.7291 & $0.2 \pm 1.6 \pm 0.3$& $1.5 \pm 0.7 \pm 0.4$& $4.9 \pm 1.0 \pm 1.1$& $19.2 \pm 1.1 \pm 3.7$\\ 
GRB120422A & 0.2826 & $3.2 \pm 0.3 \pm 0.1$& $6.4 \pm 0.5 \pm 0.3$& $14.8 \pm 0.7 \pm 0.6$& $57.5 \pm 1.1 \pm 2.1$\\ 
GRB120624B & 2.1974 & $-0.2 \pm 0.6 \pm 0.1$& \nodata & $2.9 \pm 1.4 \pm 1.0$ & $10.3 \pm 0.7 \pm 3.1$\\ 
GRB120714B & 0.3985 & $0.7 \pm 0.5 \pm 0.1$& $0.5 \pm 0.3 \pm 0.1$& $2.5 \pm 0.2 \pm 0.2$& $7.5 \pm 0.3 \pm 0.5$\\ 
GRB120722A & 0.9590 & $1.4 \pm 0.2 \pm 0.2$& $2.6 \pm 0.2 \pm 0.3$& $8.7 \pm 1.2 \pm 1.0$& $32.0 \pm 1.5 \pm 3.4$\\ 
GRB120815A & 2.3587 & \nodata & \nodata & $0.3 \pm 0.1 \pm 0.1$& $0.8 \pm 0.2 \pm 0.2$\\ 
GRB121024A & 2.3012 & \nodata & \nodata & $7.3 \pm 0.6 \pm 0.6$& $17.7 \pm 1.2 \pm 1.5$\\ 
GRB121027A & 1.7732 & \nodata & \nodata & \nodata & \nodata \\ 
% GRB 121201A added by hand during refereeing
GRB121201A & 3.3830 & $0.6 \pm 0.4 \pm 0.1$ & \nodata & \nodata & \nodata \\ 
GRB130131B & 2.5393 & \nodata & \nodata & $0.5 \pm 0.4 \pm 0.2$& \nodata \\ 
GRB130427A & 0.3401 & $-0.7 \pm 1.0 \pm 0.1$& $0.3 \pm 1.1 \pm 0.1$& $4.8 \pm 0.9 \pm 0.4$& $14.6 \pm 1.3 \pm 1.1$\\ 
GRB130701A & 1.1548 & \nodata & \nodata & $1.3 \pm 2.4 \pm 0.2$& \nodata \\ 
GRB130925A & 0.3483 & $1.2 \pm 1.0 \pm 0.1$& $5.9 \pm 1.0 \pm 0.3$& $11.7 \pm 0.7 \pm 0.6$& $54 \pm 2 \pm 3$\\ 
GRB131103A & 0.5960 & $3.1 \pm 1.1 \pm 0.6$& $6.0 \pm 1.3 \pm 1.0$& $20.2 \pm 1.1 \pm 2.8$& $51 \pm 5 \pm 8$\\ 
GRB131105A & 1.6854 & $1.0 \pm 0.6 \pm 0.3$& $0.7 \pm 0.6 \pm 0.3$& $1.7 \pm 0.5 \pm 0.5$& $10.1 \pm 0.7 \pm 2.3$\\ 
GRB131231A & 0.6427 & $1.1 \pm 0.3 \pm 0.1$& $2.4 \pm 0.3 \pm 0.2$& $3.9 \pm 0.5 \pm 0.4$& $14.1 \pm 1.1 \pm 1.4$\\ 
GRB140213A & 1.2079 & \nodata & \nodata & $0.6 \pm 0.7 \pm 0.1$& $1.2 \pm 0.5 \pm 0.1$\\ 
GRB140301A & 1.4155 & $2.1 \pm 0.6 \pm 0.2$& $1.4 \pm 0.6 \pm 0.2$& $5.0 \pm 0.6 \pm 0.5$& $31.7 \pm 1.5 \pm 2.7$\\ 
GRB140430A & 1.6019 & \nodata & \nodata & \nodata & $5.8 \pm 0.8 \pm 0.6$\\ 
GRB140506A & 0.8893 & \nodata & \nodata & \nodata & $1.2 \pm 0.4 \pm 0.2$\\ 

\hline\noalign{\smallskip}
\end{longtable}
\centering
\begin{minipage}{5.8in}
\tablefoot{
Measurements are in units of $10^{-17}$\,\erg, and are corrected for the Galactic foreground reddening. A correction for slit-loss based on broad-band photometry has been applied to the measurements as described in the text. These values thus represent host-integrated measurements except for GRB~100316D (see below). Line fluxes are not corrected for host-intrinsic extinction. The first error represents the statistical error due to photon statistics and line-flux measurement. The second error is the systematic error in the absolute flux calibration due to slit-loss and scaling to photometry. Redshifts are given in a heliocentric reference frame. No data means that either the wavelength range of the respective line is not covered, all data in that wavelength range have been excluded from the automated fitting procedure, or no meaningful constraints could be obtained for the given line.\\
$^{a}$ GRB~100316D has the lowest redshift in the sample ($z=0.0592$). The fraction of the host that is covered by the slit is too small to derive representative host-integrated measurements. These values are thus as derived from the observed spectrum and not scaled by photometry. No systematic error on the line fluxes is given in this case.}
\end{minipage}

%\end{landscape}
\end{longtab}
