\begin{longtab}
%\begin{landscape}
\begin{longtable}{c c c c c c c}
\caption{Physical properties of GRB hosts\label{tab:physprop}}\\
\hline\hline
{GRB host$^{a}$} & {Redshift} & $E_{B-V}$ & SF-tracer & SFR$^{b}$  & $\sigma$ & $Z$ \\[1.5pt]
\hline
{}         &    {}      & (mag)     &   & (\Msunyr) & (\kms) & $12+\log(\rm{O/H})$ \\[1.5pt]
\hline
\endfirsthead
\hline 
\caption{Physical properties of GRB hosts (continued)}\\
\hline\hline
{GRB host} & {Redshift} & $E_{B-V}$ & SF-tracer & SFR$^{a}$  & $\sigma$ & $Z$ \\[1.5pt]
\hline
{}         &    {}      & (mag)     &    & (\Msunyr) & (\kms) & $12+\log(\rm{O/H})$ \\[1.5pt]
\hline
\endhead
GRB050416A & 0.6542 & $0.49_{-0.11}^{+0.11}$ & \ha & $4.5_{-1.2}^{+1.6}$ & $47\pm{4}$ & $8.46_{-0.11}^{+0.11}$ \\ [1.5pt] 
GRB050525A & 0.6063 & $0.10_{-0.10}^{+0.62}$ & \ha & $0.07_{-0.05}^{+0.21}$ & $26\pm{5}$ & \nodata \\ [1.5pt] 
GRB050714B & 2.4383 & $0.21_{-0.21}^{+0.28}$ & \ha & $12.9_{-5.3}^{+14.0}$ & $35\pm{16}$ & \nodata \\ [1.5pt] 
GRB050819A & 2.5042 & $0.34_{-0.34}^{+0.84}$ & \hb & $22_{-15}^{+426}$ & $50\pm{6}$ & \nodata \\ [1.5pt] 
GRB050824 & 0.8277 & $0.00_{-0.00}^{+0.07}$ & \ha & $1.20_{-0.26}^{+0.30}$ & $48\pm{5}$ & $8.11_{-0.20}^{+0.18}$ \\ [1.5pt] 
GRB050915A & 2.5275 & $0.84_{-0.60}^{+0.61}$ & \hb & $196_{-174}^{+1563}$ & $87\pm{10}$ & \nodata \\ [1.5pt] 
GRB051001 & 2.4295 & $0.58_{-0.28}^{+0.28}$ & \ha & $110_{-59}^{+124}$ & $67\pm{8}$ & \nodata \\ [1.5pt] 
GRB051016B & 0.9358 & $0.05_{-0.05}^{+0.07}$ & \ha & $10.2_{-2.0}^{+2.6}$ & $58\pm{4}$ & $8.27_{-0.20}^{+0.15}$ \\ [1.5pt] 
GRB051022A & 0.8061 & $0.56_{-0.04}^{+0.05}$ & \ha & $60_{-10}^{+12}$ & $88\pm{5}$ & $8.49_{-0.09}^{+0.09}$ \\ [1.5pt] 
GRB051117B & 0.4805 & $0.72_{-0.24}^{+0.27}$ & \ha & $4.7_{-2.2}^{+4.9}$ & $85\pm{11}$ & $9.00_{-0.16}^{+0.16}$ \\ [1.5pt] 
GRB060204B & 2.3393 & $0.34_{-0.23}^{+0.29}$ & \ha & $78_{-34}^{+85}$ & $85\pm{8}$ & \nodata \\ [1.5pt] 
GRB060306 & 1.5597 & $0.44_{-0.40}^{+0.69}$ & \ha & $17.6_{-11.0}^{+83.6}$ & $60\pm{14}$ & \nodata \\ [1.5pt] 
GRB060604 & 2.1355 & $0.38_{-0.27}^{+0.32}$ & \ha & $7.2_{-3.6}^{+9.4}$ & $68\pm{13}$ & $8.10_{-0.35}^{+0.28}$ \\ [1.5pt] 
GRB060707 & 3.4246 & \nodata & \oii & $19.9_{-14.3}^{+48.0}$ & \nodata & \nodata \\ [1.5pt] 
GRB060719 & 1.5318 & $0.40_{-0.36}^{+0.52}$ & \ha & $7.1_{-3.9}^{+18.9}$ & $42\pm{5}$ & $8.61_{-0.24}^{+0.20}$ \\ [1.5pt] 
GRB060729 & 0.5429 & $0.71_{-0.47}^{+0.43}$ & \ha & $0.96_{-0.69}^{+2.21}$ & $66\pm{16}$ & \nodata \\ [1.5pt] 
GRB060805A & 2.3633 & $0.00_{-0.00}^{+0.16}$ & \ha & $9.0_{-2.5}^{+3.9}$ & $71\pm{7}$ & \nodata \\ [1.5pt] 
GRB060814 & 1.9223 & $0.17_{-0.17}^{+0.39}$ & \ha & $54_{-19}^{+89}$ & $132\pm{11}$ & \nodata \\ [1.5pt] 
GRB060912A & 0.9362 & $0.16_{-0.09}^{+0.10}$ & \ha & $5.1_{-1.6}^{+2.1}$ & $62\pm{5}$ & $8.61_{-0.12}^{+0.11}$ \\ [1.5pt] 
GRB060923B & 1.5094 & \nodata & \ha & $3.0_{-1.5}^{+2.9}$ & $46\pm{15}$ & \nodata \\ [1.5pt] 
GRB060926 & 3.2090 & \nodata & \oiii & $26_{-17}^{+47}$ & $60\pm{8}$ & \nodata \\ [1.5pt] 
GRB061021 & 0.3453 & $0.11_{-0.11}^{+0.20}$ & \ha & $0.05_{-0.01}^{+0.03}$ & $16.1\pm{4.8}$ & \nodata \\ [1.5pt] 
GRB061110A & 0.7578 & \nodata & \hb & $0.23_{-0.15}^{+0.38}$ & $31\pm{4}$ & \nodata \\ [1.5pt] 
GRB061202 & 2.2543 & $0.58_{-0.27}^{+0.34}$ & \ha & $43_{-22}^{+60}$ & $64\pm{7}$ & \nodata \\ [1.5pt] 
GRB070103 & 2.6208 & $0.00_{-0.00}^{+0.50}$ & \hb & $43_{-17}^{+162}$ & $124\pm{30}$ & \nodata \\ [1.5pt] 
GRB070110 & 2.3523 & $0.00_{-0.00}^{+0.38}$ & \ha & $8.9_{-2.8}^{+10.9}$ & $24\pm{4}$ & \nodata \\ [1.5pt] 
GRB070129 & 2.3384 & $0.17_{-0.17}^{+0.35}$ & \ha & $20_{-7}^{+28}$ & $76\pm{11}$ & \nodata \\ [1.5pt] 
GRB070224 & 1.9922 & \nodata & \oiii & $3.2_{-2.3}^{+6.5}$ & $37\pm{9}$ & \nodata \\ [1.5pt] 
GRB070306 & 1.4965 & $0.43_{-0.07}^{+0.08}$ & \ha & $101_{-18}^{+24}$ & $121\pm{55}$ & $8.54_{-0.09}^{+0.09}$ \\ [1.5pt] 
GRB070318 & 0.8401 & $0.15_{-0.14}^{+0.16}$ & \ha & $0.79_{-0.24}^{+0.44}$ & $53\pm{5}$ & \nodata \\ [1.5pt] 
GRB070328 & 2.0627 & $0.16_{-0.16}^{+0.79}$ & \hb & $8.4_{-4.2}^{+130.7}$ & $93\pm{14}$ & \nodata \\ [1.5pt] 
GRB070419B & 1.9586 & $0.56_{-0.30}^{+0.39}$ & \ha & $21_{-11}^{+35}$ & $86\pm{10}$ & \nodata \\ [1.5pt] 
GRB070521 & 2.0865 & \nodata & \ha & $26_{-17}^{+34}$ & $249\pm{108}$ & \nodata \\ [1.5pt] 
GRB070802 & 2.4538 & $0.31_{-0.12}^{+0.12}$ & \ha & $24_{-8}^{+11}$ & $57\pm{5}$ & \nodata \\ [1.5pt] 
GRB071021 & 2.4515 & $0.19_{-0.17}^{+0.16}$ & \ha & $32_{-12}^{+20}$ & $100\pm{17}$ & \nodata \\ [1.5pt] 
GRB080207 & 2.0856 & $0.66_{-0.25}^{+0.28}$ & \ha & $77_{-38}^{+86}$ & $136\pm{18}$ & $8.74_{-0.15}^{+0.15}$ \\ [1.5pt] 
GRB080413B & 1.1012 & $0.43_{-0.32}^{+0.30}$ & \ha & $2.1_{-1.2}^{+3.1}$ & $39\pm{5}$ & $8.29_{-0.30}^{+0.32}$ \\ [1.5pt] 
GRB080602 & 1.8204 & $0.58_{-0.26}^{+0.29}$ & \ha & $125_{-65}^{+145}$ & $91\pm{13}$ & \nodata \\ [1.5pt] 
GRB080605 & 1.6408 & $0.26_{-0.10}^{+0.11}$ & \ha & $47_{-12}^{+17}$ & $80\pm{6}$ & $8.54_{-0.09}^{+0.09}$ \\ [1.5pt] 
GRB080804 & 2.2059 & $0.38_{-0.35}^{+0.51}$ & \ha & $15.2_{-8.7}^{+41.2}$ & $50\pm{9}$ & \nodata \\ [1.5pt] 
GRB080805 & 1.5052 & $0.78_{-0.31}^{+0.39}$ & \ha & $45_{-26}^{+79}$ & $54\pm{12}$ & $8.49_{-0.14}^{+0.13}$ \\ [1.5pt] 
GRB081109 & 0.9785 & $0.36_{-0.10}^{+0.11}$ & \ha & $11.8_{-2.9}^{+4.1}$ & $108\pm{6}$ & $8.75_{-0.09}^{+0.09}$ \\ [1.5pt] 
GRB081210 & 2.0631 & $0.13_{-0.13}^{+0.61}$ & \hb & $15.3_{-7.0}^{+111.7}$ & $118\pm{12}$ & \nodata \\ [1.5pt] 
GRB081221 & 2.2590 & $0.31_{-0.31}^{+0.55}$ & \ha & $35_{-22}^{+106}$ & $93\pm{12}$ & \nodata \\ [1.5pt] 
GRB090113 & 1.7494 & $0.01_{-0.01}^{+0.22}$ & \ha & $17.9_{-4.8}^{+10.1}$ & $70\pm{9}$ & \nodata \\ [1.5pt] 
GRB090201 & 2.1000 & $0.11_{-0.11}^{+0.19}$ & \ha & $48_{-14}^{+30}$ & $171\pm{12}$ & \nodata \\ [1.5pt] 
GRB090323 & 3.5832 & \nodata & \oii & $24_{-17}^{+53}$ & $60\pm{13}$ & \nodata \\ [1.5pt] 
GRB090407 & 1.4478 & $0.69_{-0.26}^{+0.34}$ & \ha & $13.8_{-6.7}^{+18.8}$ & $109\pm{8}$ & $8.85_{-0.13}^{+0.13}$ \\ [1.5pt] 
GRB090926B & 1.2427 & $0.63_{-0.18}^{+0.20}$ & \ha & $26_{-11}^{+19}$ & $65\pm{4}$ & $8.34_{-0.17}^{+0.15}$ \\ [1.5pt] 
GRB091018 & 0.9710 & $0.06_{-0.06}^{+0.56}$ & \ha & $1.29_{-0.32}^{+3.46}$ & $57\pm{10}$ & $8.78_{-0.19}^{+0.18}$ \\ [1.5pt] 
GRB091127 & 0.4904 & $0.16_{-0.08}^{+0.09}$ & \ha & $0.37_{-0.07}^{+0.10}$ & $30\pm{5}$ & $8.07_{-0.20}^{+0.18}$ \\ [1.5pt] 
GRB100418A & 0.6235 & $0.17_{-0.07}^{+0.06}$ & \ha & $4.2_{-0.8}^{+1.0}$ & $56\pm{4}$ & $8.52_{-0.10}^{+0.10}$ \\ [1.5pt] 
GRB100424A & 2.4656 & $0.13_{-0.13}^{+0.18}$ & \hb & $21_{-8}^{+20}$ & $87\pm{5}$ & $7.93_{-0.18}^{+0.25}$ \\ [1.5pt] 
GRB100508A & 0.5201 & $0.29_{-0.09}^{+0.09}$ & \ha & $2.6_{-0.5}^{+0.7}$ & $80\pm{11}$ & $8.68_{-0.10}^{+0.10}$ \\ [1.5pt] 
GRB100606A & 1.5545 & $0.05_{-0.05}^{+0.60}$ & \ha & $4.9_{-1.8}^{+12.9}$ & $107\pm{36}$ & $8.71_{-0.21}^{+0.19}$ \\ [1.5pt] 
GRB100615A & 1.3978 & $0.48_{-0.28}^{+0.38}$ & \ha & $8.6_{-4.4}^{+13.9}$ & $45\pm{5}$ & $8.40_{-0.13}^{+0.12}$ \\ [1.5pt] 
GRB100621A & 0.5426 & $0.05_{-0.03}^{+0.03}$ & \ha & $8.7_{-0.8}^{+0.8}$ & $82\pm{4}$ & $8.52_{-0.10}^{+0.10}$ \\ [1.5pt] 
GRB100724A & 1.2890 & $0.24_{-0.24}^{+0.37}$ & \ha & $3.2_{-1.4}^{+5.1}$ & $58\pm{7}$ & \nodata \\ [1.5pt] 
GRB100728A & 1.5670 & $0.23_{-0.23}^{+0.69}$ & \ha & $14.5_{-8.0}^{+60.6}$ & $57\pm{8}$ & \nodata \\ [1.5pt] 
GRB100814A & 1.4392 & $0.08_{-0.08}^{+0.26}$ & \ha & $3.2_{-0.7}^{+2.9}$ & $31\pm{5}$ & \nodata \\ [1.5pt] 
GRB100816A & 0.8048 & $1.32_{-0.22}^{+0.24}$ & \ha & $58_{-26}^{+51}$ & $111\pm{30}$ & $8.75_{-0.18}^{+0.16}$ \\ [1.5pt] 
GRB110808A & 1.3490 & $0.30_{-0.25}^{+0.34}$ & \ha & $8.3_{-3.6}^{+11.3}$ & $40\pm{4}$ & $7.93_{-0.23}^{+0.31}$ \\ [1.5pt] 
GRB110818A & 3.3609 & \nodata & \hb & $44_{-26}^{+62}$ & $89\pm{8}$ & $8.25_{-0.25}^{+0.17}$ \\ [1.5pt] 
GRB110918A & 0.9843 & $0.35_{-0.31}^{+0.31}$ & \ha & $23_{-11}^{+28}$ & $126\pm{18}$ & $8.93_{-0.11}^{+0.11}$ \\ [1.5pt] 
GRB111123A & 3.1513 & \nodata & \oii & $77_{-52}^{+163}$ & $135\pm{21}$ & $8.01_{-0.28}^{+0.28}$ \\ [1.5pt] 
GRB111129A & 1.0796 & \nodata & \oii & $5.1_{-3.4}^{+10.8}$ & $117\pm{35}$ & \nodata \\ [1.5pt] 
GRB111209A & 0.6770 & $0.16_{-0.16}^{+0.20}$ & \ha & $0.35_{-0.13}^{+0.26}$ & $35\pm{5}$ & \nodata \\ [1.5pt] 
GRB111211A & 0.4786 & $0.00_{-0.00}^{+0.57}$ & \ha & $0.12_{-0.03}^{+0.29}$ & $38\pm{8}$ & \nodata \\ [1.5pt] 
GRB111228A & 0.7164 & \nodata & \hb & $0.32_{-0.22}^{+0.56}$ & $19.7\pm{5.5}$ & \nodata \\ [1.5pt] 
GRB120118B & 2.9428 & $0.00_{-0.00}^{+0.16}$ & \hb & $28_{-11}^{+21}$ & $193\pm{8}$ & $7.89_{-0.17}^{+0.23}$ \\ [1.5pt] 
GRB120119A & 1.7291 & $0.35_{-0.14}^{+0.16}$ & \ha & $43_{-14}^{+24}$ & $104\pm{17}$ & $8.60_{-0.14}^{+0.14}$ \\ [1.5pt] 
GRB120422A & 0.2826 & $0.27_{-0.03}^{+0.03}$ & \ha & $1.38_{-0.12}^{+0.13}$ & $25\pm{4}$ & $8.39_{-0.09}^{+0.09}$ \\ [1.5pt] 
GRB120624B & 2.1974 & $0.21_{-0.21}^{+0.50}$ & \ha & $30_{-13}^{+73}$ & $77\pm{6}$ & $8.43_{-0.27}^{+0.20}$ \\ [1.5pt] 
GRB120714B & 0.3985 & $0.10_{-0.08}^{+0.08}$ & \ha & $0.27_{-0.05}^{+0.07}$ & $34\pm{4}$ & $8.39_{-0.11}^{+0.11}$ \\ [1.5pt] 
GRB120722A & 0.9590 & $0.46_{-0.05}^{+0.05}$ & \ha & $22_{-4}^{+4}$ & $56\pm{4}$ & $8.48_{-0.10}^{+0.10}$ \\ [1.5pt] 
GRB120815A & 2.3587 & $0.06_{-0.06}^{+0.34}$ & \ha & $2.3_{-1.0}^{+2.7}$ & $28\pm{5}$ & \nodata \\ [1.5pt] 
GRB121024A & 2.3012 & $0.00_{-0.00}^{+0.12}$ & \ha & $37_{-4}^{+4}$ & $88\pm{4}$ & $8.41_{-0.12}^{+0.11}$ \\ [1.5pt] 
GRB121027A & 1.7732 & \nodata & \oiii & $24_{-15}^{+41}$ & $119\pm{75}$ & \nodata \\ [1.5pt] 
GRB121201A & 3.3830 & \nodata & \oii & $30_{-21}^{+68}$ & $86\pm{17}$ & \nodata \\ [1.5pt] 
GRB130131B & 2.5393 & \nodata & \oiii & $8.0_{-5.0}^{+13.4}$ & $73\pm{29}$ & \nodata \\ [1.5pt] 
GRB130427A & 0.3401 & $0.06_{-0.06}^{+0.19}$ & \ha & $0.34_{-0.06}^{+0.20}$ & $40\pm{5}$ & $8.57_{-0.13}^{+0.12}$ \\ [1.5pt] 
GRB130701A & 1.1548 & \nodata & \oii & $0.78_{-0.60}^{+2.03}$ & $82\pm{42}$ & \nodata \\ [1.5pt] 
GRB130925A & 0.3483 & $0.41_{-0.06}^{+0.06}$ & \ha & $2.9_{-0.4}^{+0.5}$ & $49\pm{5}$ & $8.73_{-0.08}^{+0.08}$ \\ [1.5pt] 
GRB131103A & 0.5960 & $0.06_{-0.06}^{+0.07}$ & \ha & $4.4_{-0.9}^{+1.2}$ & $87\pm{7}$ & $8.48_{-0.12}^{+0.10}$ \\ [1.5pt] 
GRB131105A & 1.6854 & $0.53_{-0.18}^{+0.21}$ & \ha & $31_{-13}^{+25}$ & $52\pm{11}$ & $8.61_{-0.20}^{+0.17}$ \\ [1.5pt] 
GRB131231A & 0.6427 & $0.02_{-0.02}^{+0.08}$ & \ha & $1.38_{-0.20}^{+0.28}$ & $33\pm{4}$ & $8.45_{-0.12}^{+0.11}$ \\ [1.5pt] 
GRB140213A & 1.2079 & $0.06_{-0.06}^{+0.72}$ & \ha & $0.72_{-0.34}^{+2.65}$ & $34\pm{14}$ & \nodata \\ [1.5pt] 
GRB140301A & 1.4155 & $0.75_{-0.10}^{+0.11}$ & \ha & $106_{-25}^{+36}$ & $117\pm{6}$ & $8.89_{-0.09}^{+0.09}$ \\ [1.5pt] 
GRB140430A & 1.6019 & \nodata & \ha & $8.5_{-3.8}^{+7.1}$ & $40\pm{7}$ & $8.67_{-0.19}^{+0.18}$ \\ [1.5pt] 
GRB140506A & 0.8893 & \nodata & \ha & $0.35_{-0.19}^{+0.35}$ & $61\pm{9}$ & \nodata \\ [1.5pt] 

\hline\noalign{\smallskip}
\end{longtable}
\centering
\begin{minipage}{5.3in}
\tablefoot{
$^{a}$ The physical parameters presented here are integrated and thus averaged over the entire galaxy. We do not perform a resolved analysis (neither spatially, nor in velocity space). In some cases \citep[e.g., GRB~120422A,][]{2014A&A...566A.102S}, a spatially resolved analysis leads to somewhat different results and interpretation of the galaxy properties.
$^{b}$ The quoted error on SFR is logarithmic because it contains the error in the dust correction. The derived SFR also has a lower limit because of the physical condition that $A_V > 0$\,mag. This lower limit is given by SFR$_{\rm min} = 4.8 \times \rm{F_{{H}\alpha, 42}}$.\\}
\end{minipage}

%\end{landscape}
\end{longtab}
